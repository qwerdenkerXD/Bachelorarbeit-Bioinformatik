\section{Einleitung} % (fold)
    \label{sec:einleitung}
    \dots Wissenschaftlicher Kontext, zufällige Ähnlichkeit in Alignments \dots

    Diese Bachelorarbeit, die auf dem Projekt \protfin\ aufbaut, stellt sich diesem Problem zufälliger Ähnlichkeit und beschäftigt sich daher mit der Frage, ob es möglich ist, funktionsähnliche Proteine über ihre physikalischen Eigenschaften zu identifizieren, anstelle der lediglichen Buchstaben ihrer Aminosäuren, und ob das die Problematik umgeht.

    Eine Grundlage hierfür bildet die Arbeit von Akinori Kidera \textit{et. al.}, welcher in seiner Forschungsgruppe mittels statistischer Faktorenanalyse 188 physikalische Eigenschaften der 20 natürlich vorkommenden Aminosäuren auf lediglich 10 sogenannte Kidera-Faktoren reduziert hat, die zusammen all diese Eigenschaften am besten erklären \vgl{kidera}. So ist beispielsweise die Hydrophobizität ein ebensolcher Faktor, da diese mit vielen anderen Eigenschaften stark in Korrelation steht.

    Der Einfluss eines jeden Faktors in einer Aminosäure lässt sich numerisch darstellen, sodass eine Aminosäuresequenz in 10 Vektoren übersetzt werden kann, welche nun ein statistisch auswertbares Abbild der physikalischen Struktur des Proteins erzeugen.

    Der Algorithmus für die Analyse dieser Struktur ist von SHAZAM inspiriert, einer Anwendung, die Musiktitel anhand kürzester Tonaufnahmen identifiziert, selbst wenn diese Störgeräusche aufweisen. Basis hierfür stellt die Short Time Fourier Transformation (STFT) dar, welche in dem musikalischen Spektrum intervallweise periodisch auftretende Signale analysiert, wodurch auch die Störgeräusche eine geringe Relevanz haben. Damit die Musikerkennung funktioniert, wird nun vorher eine Datenbank erstellt, welche die Periodizitäten der Eingabesongs mittels Hashing effizient auffindbar abspeichert, sodass der Abgleich mit einer Tonaufnahme sehr schnell und korrekt abläuft \vgl{wang}.

    Für die Anwendung auf Proteine werden statt des musikalischen Spektrums die numerischen Vektoren der Aminosäuresequenzen verwendet. Nun gibt es in \protfin\ zwei verschiedene Anwendungsansätze:
    \begin{enumerate}
        \item \textbf{Single-Protein:}\ \ Als Eingabe erfolgt eine einzelne Aminosäuresequenz, für die das best passende Protein gesucht wird. Je mehr Übereinstimmung herrscht, desto funktionsähnlicher sollte es sein.
        \item \textbf{Family-Matching:}\ \ Als Eingabe erfolgt eine Proteinfamilie. Die Periodizitäten, in denen sich alle Mitglieder dieser Familie ähneln, die also spezifisch für die Familie sind, werden verwendet, um Proteine zu finden, die auch in die Familie passen.
    \end{enumerate}
    Um den Algorithmus für beide Ansätze auf die vergleichsweise kurzen Sequenzen von wenigen 100 Elementen abzustimmen (ein solcher Vektor für eine Sekunde Musik hätte etwa 40.000 Elemente), wurde in vorangegangenen Experimenten versucht, die Intervall-/Fenstergröße, die Überlappung zwischen diesen Fenstern bei der STFT zu optimieren, beziehungsweise auch die Anzahl gewählter Frequenzen, deren Amplituden auf Periodizität hindeuten. Hierbei zeigte sich allerdings eine schlechte Performanz hinsichtlich Speicher- und Laufzeitkomplexität. Die Wahl der Frequenzen der STFT Fenster und deren Abspeicherung müssen folglich noch verbessert werden.

    Hierzu werden in dieser Arbeit mehrere Experimente angegangen.
% section einleitung (end)