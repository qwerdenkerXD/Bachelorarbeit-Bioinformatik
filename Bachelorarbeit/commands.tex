\newcommand{\Author}{Franz-Eric Sill}
\newcommand{\Date}{26.08.2024}
\newcommand{\Title}{Proteinerkennung auf Basis physikalischer Eigenschaften mittels Fourier-Transformation}
\newcommand{\Subtitle}{}
\newcommand{\Purpose}{Bachelorarbeit}
\newcommand{\protfin}{prot-fin}
\newcommand{\vgl}[1]{\cite[vgl.][]{#1}}
\floatname{algorithm}{Algorithmus}
\newcommand{\algorithmautorefname}{Algorithmus}
\renewcommand{\listalgorithmname}{Algorithmenverzeichnis}
\renewcommand{\algorithmicrequire}{\textbf{Vorbedingung}}
\renewcommand{\algorithmicensure}{\textbf{Nachbedingung}}
\newcommand{\linenoautorefname}{Line}

\newcommand{\MyRef}[1]{\autoref{#1} (\autopageref{#1})}
\newcommand{\Anhang}[1]{\texttt{#1} im Anhang (\autopageref{sec:anhang})}

% referencing algorithm lines, the custom counter is necessary when referencing same line numbers of different algorithms
\newcounter{algline}[algorithm]
% \renewcommand{\thealgline}{Zeile \arabic{algline}}
\newcommand{\alglineautorefname}{Zeile}
\newcommand{\AlgLineLabel}[1]{
  \setcounter{algline}{\value{ALG@line}}  % set to current line
  \addtocounter{algline}{-1}  % Adjust line number
  \refstepcounter{algline}%  add reference for hyperref
  \label{#1}
}
\algnewcommand\algorithmicforeach{\textbf{for each}}
\algdef{S}[FOR]{ForEach}[2]{\algorithmicforeach\ #1\ \textbf{in}\ #2\ \algorithmicdo}
\newcommand{\RefAlgLine}[2]{\autoref{#1:#2} von \autoref{#1}}
\newcommand{\sminus}{\scalebox{0.8}[1]{-}}

\newcommand{\centerIt}[1]{\hspace*{\fill} #1 \hspace*{\fill}}  % used in table headers

\newcommand{\PhantomSubSub}[2][]{
            \phantomsection
            \ifx{\relax#1\relax}{}
            \else
            \label{#1}%
            \fi
            % \ifx\relax#1\relax \label{#1} \fi
            % \addtocounter{subsubsection}{1}
            % \addcontentsline{toc}{subsubsection}{\protect\numberline{\thesubsubsection} #1}}
            \addcontentsline{toc}{subsubsection}{#2}}

\newcounter{experiment}
% \renewcommand{\theexperiment}{\arabic{experiment}}
\newenvironment{experiment}[1]{\addtocounter{experiment}{1}\subsection{\acl{Exp.} \theexperiment: #1}\addtocounter{experiment}{-1}\refstepcounter{experiment}}{}
\newcommand{\experimentautorefname}{\ac{Exp.}}
\newcommand{\Exp}[1]{Exp.~\ref*{#1}}  % for referencing in captions that appear before introduction like the lof
% \newcommand{\Exp}[1]{\hyperref[#1]{\acs{Exp.}~\ref{#1}}}  % with links inside lof to the experiments

\definecolor{folderbg}{RGB}{124,166,198}
\definecolor{folderborder}{RGB}{110,144,169}

\def\Size{4pt}

\newcommand\myfolder[2][fblue]{%
    \begin{tikzpicture}[overlay]
        \begin{scope}
            \filldraw[draw=folderborder,top color=folderbg!50,bottom color=folderbg]
              (-1.05*\Size,0.2\Size+8pt) rectangle ++(.75*\Size,-0.2\Size-2pt);  
            \filldraw[draw=folderborder,top color=folderbg!50,bottom color=folderbg]
              (-1.15*\Size,-\Size+3pt) rectangle (1.15*\Size,\Size+3pt);
        \end{scope}  
    \end{tikzpicture}%
    \ttfamily\ \ #2%
}

\newcommand{\myRequire}[1]{
    \Require \begin{varwidth}[t]{\textwidth}
        #1
    \end{varwidth}
}

\newcommand{\myEnsure}[1]{
    \Ensure \begin{varwidth}[t]{\textwidth}
        #1
    \end{varwidth}
}

\newenvironment{dirtree}[1]{
    \begin{forest}
        for tree={
          grow'=0,
          child anchor=west,
          parent anchor=south,
          anchor=west,
          calign=first,
          inner xsep=7pt,
          inner ysep=.6pt,
          font=\ttfamily,
          edge path={
            \noexpand\path [draw, \forestoption{edge}]
            (!u.south west) +(7.5pt,0) |- (.child anchor)\forestoption{edge label};
          },
          before typesetting nodes={
            if n=1
              {insert before={[,phantom]}}
              {}
          },
          fit=band,
          before computing xy={l=25pt},
        }
        #1
    \end{forest}
}

\renewcommand{\maketitle}{
  \begin{titlepage}
      \begin{center}
          \begin{figure}
              \centering
              \includegraphics[width=1\textwidth]{unilogo.png}
          \end{figure}
          \Large
          Technische Hochschule Bingen\\
          Fachbereich 2 $-$ Technik, Informatik und Wirtschaft\\
          Angewandte Bioinformatik (B.\ Sc.)\\
          \vspace{0.5cm} 
          
          \Huge
          \textbf{\Title}\\\hspace*{\fill}\par
          \Large
          \textbf{\Subtitle}
          \vfill
          
          \LARGE
          \Purpose\\
          abgegeben am: \Date\\
          von: \Author\\
          \vspace{0.5cm}
          Betreuer: Prof.\ Dr.\ Asis Hallab\\
      \end{center}

  \end{titlepage}
}

\newcommand{\declaration}[2]{
    \leavevmode%
    \vfill\noindent
    \begin{center}
      % Declaration Title
      \textbf{#1}
    \end{center}
    #2

    \vspace*{4em}\noindent
    \hfill%
    \begin{tabular}[t]{c}
      \rule{10em}{0.4pt}\\ Unterschrift
    \end{tabular}%
    \hfill%
    \begin{tabular}[t]{c}
      \rule{10em}{0.4pt}\\ Ort und Datum
    \end{tabular}%
    \hfill\strut\par}