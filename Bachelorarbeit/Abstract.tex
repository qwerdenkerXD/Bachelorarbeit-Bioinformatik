\phantomsection  % important for correct reference in toc
\label{sec:abstract}
\addcontentsline{toc}{section}{Abstract}
    \begin{center}
      \textbf{Zusammenfassung}
    \end{center}
    Für die Bestimmung funktionaler Ähnlichkeit zwischen Proteinen haben sich in der Bioinformatik Alignments der Aminosäuresequenzen bewährt. Ein Problem, das schon in den Anfängen dieser Methode entdeckt wurde, ist die evolutionäre Distanz, die es erschwert, bei zu großer Entfernung Homologien signifikant von zufälliger Ähnlichkeit zu unterscheiden. Es wurde die Möglichkeit untersucht, in Multiplen Sequenzalignments weitere Sequenzen hinzuzuziehen, um von stark konservierten Sequenzabschnitten zu profitieren und dem Problem entgegenzuwirken. Heute bietet Künstliche Intelligenz auch eine ganz neue Option, dieses Thema anzugehen. Eine weitere Alternative soll \protfin\ bieten. Das Konzept dieses Projekts ist, inspiriert von der Musikerkennungsanwendung SHAZAM, die Proteinsequenzen in den physikalischen Eigenschaften der Aminosäuren mit der Fouriertransformation spektral zu analysieren und dieses Spektrum über ein Hashing-Verfahren unter verschiedenen Proteinen vergleichbar zu machen. In der bisherigen Entwicklung von \protfin\ wurden lediglich Proteine anhand ihrer Spektren identifiziert, was allerdings mit einer hohen Speicherkomplexität einherging. Im Rahmen dieser Arbeit wurden erfolgreich diverse Experimente durchgeführt, um den Speicherbedarf zu senken, und begonnen, die Identifikation funktionaler Ähnlichkeit zu erreichen, sodass in der zukünftigen Entwicklung der Methode der Fokus auf Letzteres gelegt werden kann.
\vspace{4em}
    \begin{otherlanguage}{english}
        \begin{center}
          \textbf{Abstract}
        \end{center}
        In bioinformatics, alignments of amino acid sequences have proven effective for determining functional similarity between proteins. A problem identified early in this method is the evolutionary distance, which makes it difficult to distinguish homologies from random similarity when the distance is too great. The possibility of including additional sequences in multiple sequence alignments to benefit from highly conserved sequence regions and counteract this problem has been investigated. Today, artificial intelligence also offers a completely new option to address this issue. Another alternative is proposed by \protfin. The concept of this project, inspired by the music recognition application SHAZAM, is to spectrally analyze protein sequences in the physical properties of amino acids using Fourier transformation and make this spectrum comparable across different proteins through a hashing process. In the previous development of \protfin, proteins were identified solely based on their spectra, which, however, involved high storage complexity. In this work, various experiments were successfully conducted to reduce memory requirements and initiate the identification of functional similarity, so that future development of the method can focus on the latter.
    \end{otherlanguage}